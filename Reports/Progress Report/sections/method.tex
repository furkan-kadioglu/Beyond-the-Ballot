\section{Method}

\subsection*{Rationale for Survey Imputation with LLMs}

There are many advanced methods for survey imputations. However, we aim to explore whether Large Language Models (LLMs) can provide more insights into voters' thought processes while imputing their survey responses. If an LLM can predict party affiliations based on given statements, it might reveal aspects of human reasoning that other survey imputation methods cannot, especially since traditional methods do not inquire about the rationale behind choices.

\subsection*{Prompt Engineering}

We can consider various potential directions such as fine-tuning, instruction tuning, representation engineering, and prompt engineering. As a basic initial step, we focus on prompt engineering to maximize the potential of our model by ensuring the quality of prompts.

\subsection*{Methodology}

Our study uses survey responses from the European Social Survey (ESS) related to political interests and socio-political orientations to predict party affiliations within the same survey. We employ two models: the first model processes survey responses for input questions and generates \textit{Statements}, which are normal human-like speeches rather than discrete survey numbers. The second model uses these statements to predict party affiliations by analyzing what someone who made those statements might think about the target survey question. We evaluate the performance by employing accuracy metrics, using ESS data as the gold standard.

\subsection*{Model Choice}

For our research, LLaMA 2 is the optimal choice due to its advanced features and suitability for handling complex natural language tasks. The ESS data encompasses a wide range of social variables, including political interest, socio-political orientations, and party allegiance, which are crucial for our study on predicting political affiliations. LLaMA 2's transformer architecture and fine-tuning with reinforcement learning from human feedback (RLHF) enable it to generate coherent, contextually appropriate text from open-ended survey responses. This aligns well with our need to convert survey responses into natural statements and then predict political orientations based on these statements. Furthermore, LLaMA 2's flexibility in fine-tuning on various platforms enhances its adaptability to our specific research requirements, ensuring accurate and nuanced predictions. This approach leverages the comprehensive nature of the ESS data and the advanced capabilities of LLaMA 2, as discussed in recent literature on state-of-the-art language models.